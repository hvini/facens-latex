%% abtex2-modelo-trabalho-academico.tex, v-1.9.7 laurocesar
%% Copyright 2012-2018 by abnTeX2 group at http://www.abntex.net.br/ 
%%
%% This work may be distributed and/or modified under the
%% conditions of the LaTeX Project Public License, either version 1.3
%% of this license or (at your option) any later version.
%% The latest version of this license is in
%%   http://www.latex-project.org/lppl.txt
%% and version 1.3 or later is part of all distributions of LaTeX
%% version 2005/12/01 or later.
%%
%% This work has the LPPL maintenance status `maintained'.
%% 
%% The Current Maintainer of this work is the abnTeX2 team, led
%% by Lauro César Araujo. Further information are available on 
%% http://www.abntex.net.br/
%%
%% This work consists of the files abntex2-modelo-trabalho-academico.tex,
%% abntex2-modelo-include-comandos and abntex2-modelo-references.bib
%%

% ------------------------------------------------------------------------
% ------------------------------------------------------------------------
% abnTeX2: Modelo de Trabalho Academico (tese de doutorado, dissertacao de
% mestrado e trabalhos monograficos em geral) em conformidade com 
% ABNT NBR 14724:2011: Informacao e documentacao - Trabalhos academicos -
% Apresentacao
% ------------------------------------------------------------------------
% ------------------------------------------------------------------------

\documentclass[
	% -- opções da classe memoir --
	12pt,				% tamanho da fonte
	oneside,			% para impressão em recto.
	a4paper,			% tamanho do papel.
	% -- opções da classe abntex2 --
	chapter=TITLE,		% títulos de capítulos convertidos em letras maiúsculas
	section=TITLE,		% títulos de seções convertidos em letras maiúsculas
	% -- opções do pacote babel --
	english,			% idioma adicional para hifenização
	french,				% idioma adicional para hifenização
	spanish,			% idioma adicional para hifenização
	brazil				% o último idioma é o principal do documento
	]{abntex2}

% ---
% Pacotes básicos 
% ---
\usepackage{helvet}			    % Usa a fonte Helvet			
\usepackage[T1]{fontenc}		% Selecao de codigos de fonte.
\usepackage[utf8]{inputenc}		% Codificacao do documento (conversão automática dos acentos)
\usepackage{indentfirst}		% Indenta o primeiro parágrafo de cada seção.
\usepackage{color}				% Controle das cores
\usepackage{graphicx}			% Inclusão de gráficos
\usepackage{microtype} 			% para melhorias de justificação
\usepackage{lib/unifacens}      % Adaptações as normas da UniFacens
\usepackage{pdfpages}           % para incluir pdf no documento
\usepackage{ragged2e}           % para alinhamento de texto
% ---
		
% ---
% Pacotes adicionais, usados apenas no âmbito do Modelo Canônico do abnteX2
% ---
\usepackage{lipsum}				% para geração de dummy text
% ---

% ---
% Pacotes de citações
% ---
\usepackage[brazilian,hyperpageref]{backref}	 % Paginas com as citações na bibl
\usepackage[alf]{abntex2cite}	% Citações padrão ABNT

% --- 
% CONFIGURAÇÕES DE PACOTES
% --- 

% ---
% Configurações do pacote backref
% Usado sem a opção hyperpageref de backref
\renewcommand{\backrefpagesname}{Citado na(s) página(s):~}
% Texto padrão antes do número das páginas
\renewcommand{\backref}{}
% Define os textos da citação
\renewcommand*{\backrefalt}[4]{
	\ifcase #1 %
		Nenhuma citação no texto.%
	\or
		Citado na página #2.%
	\else
		Citado #1 vezes nas páginas #2.%
	\fi}%
% Define a fonte padrão como serif (Arial)
\renewcommand{\familydefault}{\sfdefault}
% ---

\graphicspath{{./images/}}

% ---
% Informações sobre a coordenadoria
% ---
\coordenadoria{Coordenadoria de Engenharia da Computação}

% ---
% Informações sobre o trabalho
% ---
\titulo{Titulo do trabalho}
\subtitulo{Subtitulo se houver}
\autor{Nome do autor}
\integranteum{Nome do aluno 1}
\integrantedois{Nome do aluno 2}
\local{Sorocaba/SP}
\data{2021}

% ---
% Informações sobre orientador
% ---
\orientador{Nome do orientador}

% ---
% Informações sobre coorientador
% ---
\coorientador{}

% O preambulo deve conter o tipo do trabalho, o objetivo, 
% o nome da instituição e a área de concentração 
\preambulo{Trabalho de conclusão de curso apresentado ao Centro Universitário Facens como exigência parcial para obtenção do diploma de graduação em Engenharia da Computação.\\ Orientador: Prof. ---------------------------}
% ---


% ---
% Configurações de aparência do PDF final

% alterando o aspecto da cor azul
\definecolor{blue}{RGB}{41,5,195}

% informações do PDF
\makeatletter
\hypersetup{
     	%pagebackref=true,
		pdftitle={\@title}, 
		pdfauthor={\@author},
    	pdfsubject={\imprimirpreambulo},
	    pdfcreator={LaTeX with abnTeX2},
		pdfkeywords={abnt}{latex}{abntex}{abntex2}{trabalho acadêmico}, 
		colorlinks=true,       		% false: boxed links; true: colored links
    	linkcolor=blue,          	% color of internal links
    	citecolor=blue,        		% color of links to bibliography
    	filecolor=magenta,      		% color of file links
		urlcolor=blue,
		bookmarksdepth=4
}
\makeatother
% --- 

% ---
% Posiciona figuras e tabelas no topo da página quando adicionadas sozinhas
% em um página em branco. Ver https://github.com/abntex/abntex2/issues/170
\makeatletter
\setlength{\@fptop}{5pt} % Set distance from top of page to first float
\makeatother
% ---

% ---
% Possibilita criação de Quadros e Lista de quadros.
% Ver https://github.com/abntex/abntex2/issues/176
%
\newcommand{\quadroname}{Quadro}
\newcommand{\listofquadrosname}{Lista de quadros}

\newfloat[chapter]{quadro}{loq}{\quadroname}
\newlistof{listofquadros}{loq}{\listofquadrosname}
\newlistentry{quadro}{loq}{0}

% configurações para atender às regras da ABNT
\setfloatadjustment{quadro}{\centering}
\counterwithout{quadro}{chapter}
\renewcommand{\cftquadroname}{\quadroname\space} 
\renewcommand*{\cftquadroaftersnum}{\hfill--\hfill}

\setfloatlocations{quadro}{hbtp} % Ver https://github.com/abntex/abntex2/issues/176
% ---

% --- 
% Espaçamentos entre linhas e parágrafos 
% --- 

% O tamanho do parágrafo é dado por:
\setlength{\parindent}{1.3cm}

% Controle do espaçamento entre um parágrafo e outro:
\setlength{\parskip}{0.2cm}  % tente também \onelineskip

% ---
% compila o indice
% ---
\makeindex
% ---

% ----
% Início do documento
% ----
\begin{document}

% Seleciona o idioma do documento (conforme pacotes do babel)
%\selectlanguage{english}
\selectlanguage{brazil}

% Retira espaço extra obsoleto entre as frases.
\frenchspacing 

% ----------------------------------------------------------
% ELEMENTOS PRÉ-TEXTUAIS
% ----------------------------------------------------------
% \pretextual

% ---
% Capa
% ---
\imprimircapa
% ---

% ---
% Folha de rosto
% ---
\imprimirfolhaderosto*
% ---

% ---
% Inserir a ficha bibliografica
% ---
\begin{fichacatalografica}
	\includepdf[page=1]{main/ficha_catalografica.pdf}
\end{fichacatalografica}
% ---

% ---
% Inserir folha de aprovação
% ---
\begin{folhadeaprovacao}
	\includepdf[page=1]{main/folha_aprovacao.pdf}
\end{folhadeaprovacao}
% ---

% ---
% Agradecimentos
% ---
\begin{agradecimentos}
    Os agradecimentos principais são direcionados à Gerald Weber, Miguel Frasson,
    Leslie H. Watter, Bruno Parente Lima, Flávio de Vasconcellos Corrêa, Otavio Real
    Salvador, Renato Machnievscz e todos aqueles que
    contribuíram para que a produção de trabalhos acadêmicos conforme
    as normas ABNT com \LaTeX\ fosse possível.

    Agradecimentos especiais são direcionados ao Centro de Pesquisa em Arquitetura
    da Informação da Universidade de
    Brasília (CPAI), ao grupo de usuários
    \emph{latex-br} e aos
    novos voluntários do grupo
    \emph{\abnTeX} e que contribuíram e que ainda
    contribuirão para a evolução do \abnTeX.
\end{agradecimentos}
% ---

% ---
% Epígrafe
% ---
\begin{epigrafe}
    \vspace*{\fill}
    
    \begin{flushright}
        \begin{minipage}{8cm}
            \begin{SingleSpace}
                ``Não vos amoldeis às estruturas deste mundo,
                mas transformai-vos pela renovação da mente,
                a fim de distinguir qual é a vontade de Deus:
                o que é bom, o que Lhe é agradável, o que é perfeito.``\\
                (Bíblia Sagrada, Romanos 12, 2)
            \end{SingleSpace}
        \end{minipage}
    \end{flushright}
\end{epigrafe}
% ---

% ---
% RESUMOS
% ---
\input{main/resumo.tex}
% ---

% ---
% inserir lista de ilustrações
% ---
\pdfbookmark[0]{\listfigurename}{lof}
\listoffigures*
\cleardoublepage
% ---

% ---
% inserir lista de quadros
% ---
\pdfbookmark[0]{\listofquadrosname}{loq}
\listofquadros*
\cleardoublepage
% ---

% ---
% inserir lista de tabelas
% ---
\pdfbookmark[0]{\listtablename}{lot}
\listoftables*
\cleardoublepage
% ---

% ---
% inserir lista de abreviaturas e siglas
% ---
\begin{siglas}
    \item[AGV] Automated Guided Vehicle
    \item[AMR] Autonomous Mobile Robot
    \item[GPS] Global Position System
    \item[IA] Intelig{\^e}ncia Artificial
    \item[NEAT] Neuroevolution of Augmentating Topologies
\end{siglas}

% ---

% ---
% inserir lista de símbolos
% ---
\input{main/lista_simbolos.tex}
% ---

% ---
% inserir o sumario
% ---
\pdfbookmark[0]{\contentsname}{toc}
\tableofcontents*
\cleardoublepage
% ---



% ----------------------------------------------------------
% ELEMENTOS TEXTUAIS
% ----------------------------------------------------------
\textual

% ----------------------------------------------------------
% Introdução (exemplo de capítulo sem numeração, mas presente no Sumário)
% ----------------------------------------------------------
\chapter{Introdução}
% ----------------------------------------------------------

De forma geral, esta monografia visa apresentar a implementação de um
ambiente simulado capaz de treinar um veículo autônomo que, através dos
casos de teste, consegue se adaptar ao ambiente em que se situa buscando
chegar a determinado objetivo, através de um algoritmo de neuroevolução que
possui uma topologia aumentante. Desta forma, simulações de possíveis trajetos
e adaptações de percurso podem ser rapidamente criadas e testadas sem a
necessidade de interação com um ambiente físico.

Tratando-se de veículos autônomos, é possível citar os carros sem
motorista, segundo \citeonline{arruda2015}, são veículos capazes de realizar tarefas
[...] sem a intervenção de operadores humanos, sua demanda vem crescendo
ultimamente por conta da acessibilidade que fornecem e por permitirem uma
produtividade maior ao motorista. Considerados também veículos autônomos,
existem aqueles voltados para sistemas logísticos, onde, através de mínima
interação com operadores, salvo manutenções e reprogramações, são capazes
de realizar transporte de objetos por uma linha de produção.

A solução geralmente aplicada para carros autônomos se fundamenta no
princípio da visão computacional. Esta técnica se baseia em utilizar de câmeras
para a captura de imagens que virão a servir de entrada de dados, para que o
algoritmo de direção tome as devidas decisões quanto à melhor decisão a ser
tomada para aquela situação.

O problema desta técnica se encontra justamente na lentidão de colocar
os sistemas em prática, pois é necessário que se criem diversas situações de
teste com muitas variáveis aplicáveis, como iluminação ou resolução da foto, e
considerando que a aplicação é direcionada a carros que navegarão pelas ruas,
é necessário que estes testes sejam prolongados ainda mais para que se haja
uma garantia de segurança a todos ao redor do veículo.

Ao observar veículos autônomos dentro do ambiente fabril, é possível
encontrar máquinas chamadas de Veículos Automaticamente Guiados (AGVs),
seu uso para o transporte de contêineres data de 1993 em um terminal de
contêineres de Rotterdan \cite{saputra2021}. Estes possuem a finalidade
de transportar materiais e produtos pela fábrica, através de sensores, são
capazes de identificar a distância necessária para a movimentação do garfo
quando aplicados a uma empilhadeira ou a distância de segurança para evitar
acidentes. A sua trajetória desses veículos é geralmente predefinida, baseando-
se na planta da fábrica e o caminho que o veículo deve percorrer, como através
de faixas magnéticas, por exemplo. Sendo assim, estes dispositivos falham
quando encontram obstáculos em seu caminho, como um pallet ou caixa que
manterão sua movimentação impedida.

Tratando-se de aprendizado de máquina, existem diversas técnicas que
podem ser empregadas para se chegar a resultados simulados, estas, focadas
na modelagem da estrutura de aprendizado podem ser divididas entre,
supervisionado, não supervisionado e por reforço. A aplicação de cada método
deve ser indicada de acordo com a necessidade da aplicação, e, analisando a
proposta do projeto presente nesta monografia, é possível analisar qual solução
possui a aplicação mais justificável. Em uma observação imediata, o
aprendizado por máquina supervisionado (aquele em que se sabe quais são as
possíveis entradas e saídas equivalentes a elas) se torna inviável, tal que a
aplicação exige que o algoritmo calcule em tempo real a intermissão de possíveis
objetos em seu trajeto, sendo assim impossível definir uma saída previamente.

Este projeto procura explorar um tipo diferente de aprendizado, chamado
de aprendizado por reforço. Esse tipo de aprendizado tem a característica de
não necessitar de supervisão para o treinamento do modelo. Ou seja, não se faz
necessário treinamento em campo ou uma requisição de grande quantidade de
dados para a etapa de treinamento. Como entradas para o modelo serão
utilizados sensores de distância, em vez dos habituais sistemas de visão
computacional. O experimento será feito dentro de um ambiente virtual, com a
representação do carro e trajetos gerados aleatoriamente para treinamento.

O método de aprendizado por reforço faz necessário menos investimentos
antecipados, pois os dados de treinamento serão obtidos por um quociente de
performance ao invés de dados rotulados para retroalimentação. Além disso, a
utilização de sensores diminui significativamente o requerimento de poder
computacional para o sistema comparado com processamento de imagens, e o
mesmo pode ser dito quanto ao algoritmo de aprendizado na etapa de testes.

O aprendizado se dará pelo algoritmo NEAT (Neuroevolution of
Augmenting Topologies). Este algoritmo se inicia com uma rede neural com
apenas as entradas e saídas definidas. Por meio de um Algoritmo Genético, o
algoritmo evoluirá a topologia da rede neural de forma a melhorar o desempenho
do carro (neste caso, conseguir dirigir por mais tempo sem colisões).

Por exemplo, a princípio o algoritmo gera diversos carros com topologias
aleatórias da rede neural. Os carros que percorrerem a maior distância sem
colisões serão selecionados para reprodução - seus genes serão combinados
com outros carros de performance alta, além de também ocorrerem mutações
para o descobrimento de novos comportamentos. Este processo criará uma nova
geração de carros, que passará pelos mesmos testes, e o processo se repetirá.

Deste modo, espera-se atingir um estado de otimização em que se faz
possível o alcance da trajetória especificada sem colisões, e preferivelmente em
um caminho próximo do mais otimizado.

Apesar da elevada facilidade introduzida com esse método, a perda de
flexibilidade na utilização pode ser algo que geraria necessidade de um sistema
mais robusto como o de visão computacional. Um outro contraponto seria a
necessidade de um ambiente mais controlado, para certificar um funcionamento
correto dos sensores no ambiente e nos possíveis obstáculos. Mesmo com tais
observações, existe margem para crer que o método proposto se faz uma
alternativa interessante em aplicações em que as condições de funcionamento
podem ser mapeadas previamente


% ----------------------------------------------------------
% PARTE
% ----------------------------------------------------------
\input{main/preparacao_pesquisa.tex}
\part{Referenciais teóricos}
% ----------------------------------------------------------

% ---
% Capitulo de revisão de literatura
% ---
\chapter{Lorem ipsum dolor sit amet}
% ---

% ---
\section{Aliquam vestibulum fringilla lorem}
% ---

\lipsum[1]

\lipsum[2-3]
% ---
% primeiro capitulo de Resultados
% ---
\chapter{Resultados}
% ---

% ---
\section{Vestibulum ante ipsum primis in faucibus orci luctus et ultrices
posuere cubilia Curae}
% ---

\lipsum[21-22]

% ----------------------------------------------------------
% Finaliza a parte no bookmark do PDF
% para que se inicie o bookmark na raiz
% e adiciona espaço de parte no Sumário
% ----------------------------------------------------------
\phantompart

% ---
% Conclusão
% ---
\chapter{Considerações finais}

\textbf{***Paralelizacao e desempenho vs deep q learning; predominancia de carros andando em circulo***}

\lipsum[31-33]


% ----------------------------------------------------------
% ELEMENTOS PÓS-TEXTUAIS
% ----------------------------------------------------------
\postextual
% ----------------------------------------------------------

% ----------------------------------------------------------
% Referências bibliográficas
% ----------------------------------------------------------
\bibliography{abntex2-modelo-references}

% ----------------------------------------------------------
% Glossário
% ----------------------------------------------------------
%
% Consulte o manual da classe abntex2 para orientações sobre o glossário.
%
%\glossary

% ----------------------------------------------------------
% Apêndices
% ----------------------------------------------------------
\begin{apendicesenv}
    \chapter{Quisque libero justo}
    \lipsum[50]

    \chapter{Nullam elementum urna vel imperdiet sodales elit ipsum pharetra ligula
    ac pretium ante justo a nulla curabitur tristique arcu eu metus}
    \lipsum[55-57]
\end{apendicesenv}

% ----------------------------------------------------------
% Anexos
% ----------------------------------------------------------
\begin{anexosenv}

    \addcontentsline{toc}{chapter}{ANEXOS}
    \addtocontents{toc}{\protect\setcounter{tocdepth}{-1}}
    
    % coloca o primeiro anexo na mesma pagina do titulo do capitulo
    \chapter{Morbi ultrices rutrum lorem}
    \lipsum[30]

    \chapter{Cras non urna sed feugiat cum sociis natoque penatibus et magnis dis
    parturient montes nascetur ridiculus mus}
    \lipsum[31]

    \chapter{Fusce facilisis lacinia dui}
    \lipsum[32]

    \addtocontents{toc}{\protect\setcounter{tocdepth}{0}}

\end{anexosenv}

%---------------------------------------------------------------------
% INDICE REMISSIVO
%---------------------------------------------------------------------
\phantompart
\printindex
%---------------------------------------------------------------------

\end{document}
